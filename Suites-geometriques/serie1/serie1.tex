\documentclass[12pt]{paper}
\usepackage[french]{babel}
\usepackage[
a4paper,
margin=2cm,
nomarginpar,% We don't want any margin paragraphs
]{geometry}
\usepackage{fancyhdr}
\usepackage{array, multicol}
\usepackage{amsmath,amsfonts,amsthm,amssymb,mathtools,}
\newcolumntype{P}[1]{>{\centering\arraybackslash}p{#1}}


\usepackage{stackengine}
\newcommand\xrowht[2][0]{\addstackgap[.5\dimexpr#2\relax]{\vphantom{#1}}}

% theorems

\theoremstyle{plain}
\newtheorem{theorem}{Th\'eor\`eme}
\newtheorem*{sol}{Solution}
\theoremstyle{definition}
\newtheorem{ex}{Exercice}


% corps
\newcommand{\C}{\mathbb{C}}
\newcommand{\R}{\mathbb{R}}
\newcommand{\Rnn}{\mathbb{R}^{2n}}
\newcommand{\Z}{\mathbb{Z}}
\newcommand{\N}{\mathbb{N}}
\newcommand{\Q}{\mathbb{Q}}

% domain
\newcommand{\D}{\mathbb{D}}


% date
\usepackage{advdate}
\AdvanceDate[1]

% plots
\usepackage{pgfplots}

% for calligraphic C
\usepackage{calrsfs}

% euro
\usepackage{lmodern,textcomp}

%tablestuff
\def\arraystretch{2}
\setlength\tabcolsep{15pt}

% SOLUTION SWITCH
\newif\ifsolutions
				\solutionstrue
				\solutionsfalse

\ifsolutions
	\newcommand{\exe}[2]{
		\begin{ex} #1  \end{ex}
		\begin{sol} #2 \end{sol}
	}
\else
	\newcommand{\exe}[2]{
		\begin{ex} #1  \end{ex}
	}
	
\fi

\begin{document}
\pagestyle{fancy}
\fancyhead[L]{Première G5}
\fancyhead[C]{\textbf{Suites géométriques 1 \ifsolutions -- Solutions \fi}}
\fancyhead[R]{\today}

\exe{[Paradoxe de la frontière]

	\begin{multicols}{2}
	\begin{center}
	\begin{tikzpicture}[scale=4]
		\node (A) at (0,0) {$\bullet$};
		\node (B) at (1,0) {$\bullet$};
		\node (C) at (.5,0.866) {$\bullet$};
		
		\draw[black, thick] (A) -- (B) node[midway, below] {$1$};
		\draw[black, thick] (A) -- (C) node[midway, left] {$1$};
		\draw[black, thick] (C) -- (B) node[midway, right] {$1$};
	\end{tikzpicture}
	
	Étape $0$.
	\end{center}
	
	
	\begin{center}
	\begin{tikzpicture}[scale=4]
		\node (A) at (0,0) {$\bullet$};
		\node (B) at (1,0) {$\bullet$};
		\node (C) at (.5,0.866) {$\bullet$};
		
		\node (D) at (.33,0) {$\bullet$};
		\node (E) at (.66,0) {$\bullet$};
		\node (F) at (.5,-0.288) {$\bullet$};
		
		\node (G) at (.16,.2887) {$\bullet$};
		\node (H) at (.33,.5773) {$\bullet$};
		\node (I) at (0,0.5773) {$\bullet$};
		
		\node (J) at (.833,.2887) {$\bullet$};
		\node (K) at (.66,.5773) {$\bullet$};
		\node (L) at (1,0.5773) {$\bullet$};
		
		
		\draw[black, thick] (A) -- (D) -- (F) -- (E) -- (B) node[midway, below] {$\frac13$};
		\draw[black, thick] (A) -- (G) -- (I) -- (H) -- (C);
		\draw[black, thick] (B) -- (J) -- (L) -- (K) -- (C);
		
	\end{tikzpicture}
	
	Étape $1$.
	\end{center}
	\end{multicols}

	On considère une construction qui commence par un triangle équilatéral de côté $1$.
	À chaque étape, on divise chaque segment de la figure précédente par trois, et on remplace le tiers du milieu par un triangle équilatéral qui pointe vers l'extérieur.
	On continue ainsi indéfiniment et on se pose les questions suivantes.
	
	\begin{enumerate}
		\item Calculer le périmètre de la figure à l'étape $0$ puis à l'étape $1$.
		\item Comment le nombre de segment évolue-t-il d'une étape à l'autre ?
		\item Comment la longueur de chaque segment évolue-t-elle d'une étape à l'autre ?
		\item En déduire l'évolution du périmètre d'une étape à l'autre :
		en notant $P(n)$ le périmètre de la figure à l'étape $n=0, 1, 2, 3,\dots$, montrer que
			\[ P(n+1) = \dfrac43 P(n). \]
		\item Donner $P(2)$ et $P(3)$.
		\item Le périmètre peut-il être aussi grand qu'on le souhaite ? Par exemple, existe-t-il un rang $n$ pour lequel $P(n) \geq 1 \ 000$ ?
		\item Donner approximativement $P(100)$.
	\end{enumerate}
}{}

\exe{[Paradoxe d'Achille et de la tortue]
	Achille dispute une course avec une tortue. On suppose que les deux participants avancent à vitesse constante et que la tortue avance $10$ fois moins vite qu'Achille.
	Celui-ci décide donc de lui laisser généreusement $10$ minutes d'avance.

	En analysant la situation, Zénon décide de diviser la course en plusieurs étapes.
	À chaque étape, Achille court jusqu'au point d'où a démarré la tortue à la dernière étape.
	Il déduit que, comme la tortue avance pendant qu'Achille court, il ne pourra jamais la dépasser.

	\begin{enumerate}
		\item Vérifier les premières valeurs du tableau suivantes et le compléter.
			\begin{center}
			\begin{tabular}{|c|c|c|c|c|c|}\hline
				Étape & 0 & 1 & 2 & 3 & 4 \\ \hline
				Temps qu'Achille met pour finir l'étape (min) & 1 & 0,1 & & & \\ \hline
			\end{tabular}
			\end{center}
		\item En déduire l'évolution du temps que prend Achille d'une étape à l'autre :
		en notant $T(n)$ ce temps à l'étape $n=0, 1, 2, 3,\dots$, montrer que
			\[ T(n+1) = \dfrac1{10} T(n). \]
		\item Donner exactement $T(100)$ en écriture scientifique.
		\item On considère la somme des temps de chaque étape pour comprendre quand Achille atteindra la tortue.
			\[ S(n) = T(0) + T(1) + \dots + T(n). \]
		Vérifier les premières valeurs du tableau suivantes et le compléter.
			\begin{center}
			\begin{tabular}{|c|c|c|c|c|c|}\hline
				$n$ & 0 & 1 & 2 & 3 & 4 \\ \hline
				$S(n)$ & 1 & 1,1 & & &  \\ \hline
			\end{tabular}
			\end{center}
		\item Est-ce que $S(n)$ peut être aussi grand qu'on le souhaite ? Par exemple, existe-t-il un rang $n$ pour lequel $S(n) \geq 2$ ?
		\item En combien de temps Achille arrive-t-il à dépasser la tortue ? Donner une valeur exacte sous forme de fraction.
	\end{enumerate}
}{}

\exe{[Intérêts sur intérêts]\label{ex:2}

	À l'âge de $17$ ans une élève décide de placer $200$€ en bourse qui lui rapportent $10\%$ d'intérêts chaque année.
	Chaque année, elle replace les intérêts gagnés.
	
	On souhaite étudier l'évolution de l'argent placé chaque année après ses $17$ ans inclus.
	\begin{enumerate}
		\item Vérifier les premières valeurs du tableau suivantes et le compléter.
			\begin{center}
			\begin{tabular}{|c|c|c|c|c|c|c|}\hline
				Âge & 17 & 18 & 19 & 20 & 21 & 22 \\ \hline
				Argent placé (€) & 200 & 220 & 242 & & &  \\ \hline
			\end{tabular}
			\end{center}
		\item On appelle $A(n)$ la quantité d'argent placé à l'âge $17+n$, où $n\in \{0 ; 1 ; 2; \dots \}$.
		Décrire comment obtenir $A(n+1)$ en connaissant $A(n)$
		.%, c'est-à-dire comment obtenir la quantité d'argent placé en l'an $2017+(n+1)$ en connaissant la quantité en $2017+n$.
		
		\item Combien d'argent aura l'élève à l'âge de $50$ ans ? 
		\item Calculer $A(50)$ et interpréter le résultat.
		\item À quel âge la somme d'argent dépassera-t-elle $100 \  000$€ ?		
	\end{enumerate}
 }{}
 
 
 \exe{[Intérêts sur intérêts 2]
 
 	On reprend l'exercice \ref{ex:2} en prenant en plus en compte l'inflation, qu'on suppose constante à $2\%$ par an.
 	Cela signifie que les prix augmentent en moyenne de $2\%$ par an.
 	
 	Afin de rendre comparables les sommes d'argent dans le temps, on souhaite fixer les prix : au lieu de voir les prix comme augmentant, on voit l'euro comme se dépréciant.
 
	\begin{enumerate}
		\item Calculer l'évolution réciproque de $+2\%$. 
		C'est la diminution qu'on appliquera à l'euro chaque année.
		\item Vérifier les premières valeurs du tableau suivantes et le compléter.
			\begin{center}
			\begin{tabular}{|c|c|c|c|c|c|c|}\hline
				Année & 17 & 18 & 19 & 20 & 21 & 22 \\ \hline
				Argent placé (€, prix fixes) & 200 & 215,69 & 232,60 & & &  \\ \hline
			\end{tabular}
			\end{center}
		\item On appelle $B(n)$ la quantité d'argent placé à prix fixes en l'an $2017+n$, où $n\in \{0 ; 1 ; 2; \dots \}$.
		Décrire comment obtenir $B(n+1)$ en connaissant $B(n)$
		.%, c'est-à-dire comment obtenir la quantité d'argent placé en l'an $2017+(n+1)$ en connaissant la quantité en $2017+n$.
		
		\item Calculer $B(70)$ et interpréter le résultat.
	\end{enumerate}
 }{}

\end{document}