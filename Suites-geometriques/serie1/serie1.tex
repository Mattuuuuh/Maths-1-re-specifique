\documentclass[12pt]{paper}
\usepackage[french]{babel}
\usepackage[
a4paper,
margin=2cm,
nomarginpar,% We don't want any margin paragraphs
]{geometry}
\usepackage{fancyhdr}
\usepackage{array}
\usepackage{amsmath,amsfonts,amsthm,amssymb,mathtools,}
\newcolumntype{P}[1]{>{\centering\arraybackslash}p{#1}}


\usepackage{stackengine}
\newcommand\xrowht[2][0]{\addstackgap[.5\dimexpr#2\relax]{\vphantom{#1}}}

% theorems

\theoremstyle{plain}
\newtheorem{theorem}{Th\'eor\`eme}
\newtheorem*{sol}{Solution}
\theoremstyle{definition}
\newtheorem{ex}{Exercice}


% corps
\newcommand{\C}{\mathbb{C}}
\newcommand{\R}{\mathbb{R}}
\newcommand{\Rnn}{\mathbb{R}^{2n}}
\newcommand{\Z}{\mathbb{Z}}
\newcommand{\N}{\mathbb{N}}
\newcommand{\Q}{\mathbb{Q}}

% domain
\newcommand{\D}{\mathbb{D}}


% date
\usepackage{advdate}
\AdvanceDate[1]

% plots
\usepackage{pgfplots}

% for calligraphic C
\usepackage{calrsfs}

% euro
\usepackage{lmodern,textcomp}


% SOLUTION SWITCH
\newif\ifsolutions
				\solutionstrue
				\solutionsfalse

\ifsolutions
	\newcommand{\exe}[2]{
		\begin{ex} #1  \end{ex}
		\begin{sol} #2 \end{sol}
	}
\else
	\newcommand{\exe}[2]{
		\begin{ex} #1  \end{ex}
	}
	
\fi

\begin{document}
\pagestyle{fancy}
\fancyhead[L]{Première G5}
\fancyhead[C]{\textbf{Suites géométriques 1 \ifsolutions -- Solutions \fi}}
\fancyhead[R]{\today}

\exe{[Paradoxe de la frontière]
	

}{}

\exe{[Intérêts sur intérêts]\label{ex:2}

	À l'âge de $17$ ans, une élève décide de placer $200$€ en bourse, qui lui rapportent $10\%$ d'intérêts chaque année.
	Chaque année, elle replace les intérêts gagnés.
	
	On souhaite étudier l'évolution de l'argent placé chaque année après ses $17$ ans inclus.
	\begin{enumerate}
		\item Vérifier les premières valeurs du tableau suivantes et le compléter.
			\begin{center}
			\begin{tabular}{|c|c|c|c|c|c|}\hline
				Année & 17 & 18 & 19 & 20 & 21 \\ \hline
				Argent placé (€) & 200 & 220 & 242 & &  \\ \hline
			\end{tabular}
			\end{center}
		\item On appelle $A(n)$ la quantité d'argent placé en l'an $2017+n$, où $n\in \{0 ; 1 ; 2; \dots \}$.
		Décrire comment obtenir $A(n+1)$ en connaissant $A(n)$, c'est-à-dire comment obtenir la quantité d'argent placé en l'an $2017+(n+1)$ en connaissant la quantité en $2017+n$.
	\end{enumerate}
 }{}
 
 
 \exe{[Intérêts sur intérêts 2]
 
 	On reprend l'exercice \ref{ex:2} en prenant en plus en compte l'inflation, qu'on suppose constante à $2\%$ par an.
 	Cela signifie que les prix augmentent en moyenne de $2\%$ par an.
 	
 	Afin de rendre comparables les sommes d'argent dans le temps, on souhaite fixer les prix : au lieu de voir les prix comme augmentant, on voit l'euro comme se dépréciant.
 
	\begin{enumerate}
		\item Calculer l'évolution réciproque de $+2\%$. 
		C'est la diminution qu'on appliquera à l'euro chaque année.
		\item Vérifier les premières valeurs du tableau suivantes et le compléter.
			\begin{center}
			\begin{tabular}{|c|c|c|c|c|c|}\hline
				Année & 17 & 18 & 19 & 20 & 21 \\ \hline
				Argent placé (€, prix fixes) & 200 & 215,69 & 232,60 & &  \\ \hline
			\end{tabular}
			\end{center}
		\item On appelle $B(n)$ la quantité d'argent placé à prix fixes en l'an $2017+n$, où $n\in \{0 ; 1 ; 2; \dots \}$.
		Décrire comment obtenir $B(n+1)$ en connaissant $B(n)$, c'est-à-dire comment obtenir la quantité d'argent placé en l'an $2017+(n+1)$ en connaissant la quantité en $2017+n$.
	\end{enumerate}
 }{}

\end{document}