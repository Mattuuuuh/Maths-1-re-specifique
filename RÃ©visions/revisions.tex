\documentclass[12pt]{paper}
\usepackage[french]{babel}
\usepackage[
a4paper,
margin=2cm,
nomarginpar,% We don't want any margin paragraphs
]{geometry}
\usepackage{fancyhdr}
\usepackage{array}
\usepackage{amsmath,amsfonts,amsthm,amssymb,mathtools,}
\newcolumntype{P}[1]{>{\centering\arraybackslash}p{#1}}


\usepackage{stackengine}
\newcommand\xrowht[2][0]{\addstackgap[.5\dimexpr#2\relax]{\vphantom{#1}}}


% corps
\newcommand{\C}{\mathbb{C}}
\newcommand{\R}{\mathbb{R}}
\newcommand{\Rnn}{\mathbb{R}^{2n}}
\newcommand{\Z}{\mathbb{Z}}
\newcommand{\N}{\mathbb{N}}
\newcommand{\Q}{\mathbb{Q}}

% domain
\newcommand{\D}{\mathbb{D}}


% date
\usepackage{advdate}
\AdvanceDate[-5]

% plots
\usepackage{pgfplots}

% for calligraphic C
\usepackage{calrsfs}

% euro
\usepackage{lmodern,textcomp}

\begin{document}
\pagestyle{fancy}
\fancyhead[L]{Première G2}
\fancyhead[C]{\textbf{Exercices}}
\fancyhead[R]{\today}

\section*{Variations de fonctions}

\begin{center}
\begin{tikzpicture}[>=stealth]
	\begin{axis}[xmin = -4, xmax=4, xtick={-3, -2,-1,0,1,2, 3}, ymin=-2, ymax=5, ytick={-1,0,1,2, 3,4}, axis x line=middle, axis y line=middle, axis line style=->, xlabel={$x$}, ylabel={$y$}, grid=both]
		\addplot[no marks, blue, -] expression[domain=-2.1: 2.1, samples=100]{x^3 - 3*x + 2}
		node[pos=.86, anchor = south west]{$\mathcal{C}_f $};
	\end{axis}

\end{tikzpicture}
\end{center}

\begin{center}
\begin{tabular}{ |  P{.05\linewidth} | P{.65\linewidth} |  } 
  \hline\xrowht{10pt}
$x$  & -2 \hfill 2 \\ \hline \xrowht{60pt}
& \\ \hline
\end{tabular}
\end{center}

\section*{Droites}

\begin{center}
\begin{tabular}{ | P{.1\linewidth} | P{.1\linewidth} | P{.1\linewidth} | P{.1\linewidth} | P{.1\linewidth} |  P{.1\linewidth} | P{.1\linewidth} |  } 
  \hline\xrowht{10pt}
$x$ & -2 & -1 & 0 & 1 & 2 & 3\\ \hline \xrowht{10pt}
$f(x)$ & -5 & -3 & -1 & 1 & 3 & 5\\ \hline
\end{tabular}
\end{center}
L'image de $-1$ est \underline{  \qquad } \\
L'image de $1$ est \underline{  \qquad } \\
L'antécédent de $3$ est \underline{  \qquad }\\
L'antécédent de $1$ est \underline{  \qquad } \\
 
\noindent 
L'ordonnée à l'origine est : \\
Le coefficient directeur de la droite est : \\
L'équation de la droite est :  

\section*{Calcul littéral}

\begin{center}
\begin{tabular}{ |  P{.4\linewidth} | P{.4\linewidth} |  } 
  \hline\xrowht{10pt}
 Programme & Expression \\ \hline\xrowht{15pt}
 Choisir un nombre & $x$ \\ \hline \xrowht{15pt}
 Ajouter $3$ & \\ \hline \xrowht{15pt}
 Élever au carré & \\ \hline \xrowht{15pt}
 Multiplier par $-2$ & \\ \hline \xrowht{15pt}
 Soustraire $4$  &  \\ \hline
\end{tabular}
\end{center}
Expression finale développée : 

\newpage


\section*{Exercice en situation}

Titouan souhaite économiser afin de pouvoir s'acheter une des nouvelles consoles de jeu qui sortiront dans un an.
Parmis ces consoles, la Wintendo Qwitch coûte $320$€, la Xboux coûte $390$€ et la Pluistation coûte $550$€.

Au départ, Titouan vérifie ses économies et constate qu'il possède déjà $50$€ . Il décide alors de mettre $30 $€ de côté chaque mois pendant $1$ an.
Ainsi, à la fin du premier mois Titouan rajoute $30$€ dans sa cagnotte, il possède alors $80$€ au total.

\begin{enumerate}
	\item  Combien d'argent Titouan ajoute-t-il à ses économies chaque mois ?
	\item Donner l'argent de poche de Titouan à la fin des deux premiers mois.
	\item Combien d'argent Titouan a-t-il économisé à la fin du $12$-ème mois ?
	\item Donner l'expression de l'argent économisé à la fin du mois $n$.
	\item Combien d'argent aura-t-il économisé au bout de $5$ ans  ?
	\item En combien de temps Titouan aura-t-il économisé $1550$€? 
\end{enumerate}

\end{document}