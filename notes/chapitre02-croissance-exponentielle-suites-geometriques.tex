%!TEX encoding = UTF8
%!TEX root =notes.tex

\chapter{Croissance exponentielle}

	\section{Suites géométriques}
	
	
	\dfn{Suite}{
		Une suite $u$ est une fonction qui à tout entier naturel $n \in \N$ associe un réel
			\[ u(n) \in \R. \]
		Le \emph{terme initial} de la suite est donné par $u(0)$, et son terme de \emph{rang} $n$ est $u(n)$.
	}{}
	
	\ex{}{
		Les fonctions suivantes sont des suites données par leur rang $n$ : on connait donc leur valeur pour tout les entiers naturels.
		\begin{multicols}{2}
		\begin{enumerate}
			\item $u(n) = n+1$
			\item $v(n) = 15 + n$
			\item $\xi(n) = 3n + 1$
			\item $a(n) = n^2$
		\end{enumerate}
		\end{multicols}
		Une suite n'a pas forcément de formule générale pour tout $n$, c'est une simplement \emph{suite} de nombres réels.
	}{}
	
	
	\dfn{Suite géométrique}{
		On dit d'une suite $u$ qu'elle est arithmétique dès qu'elle vérifie, pour tout $n\in\N$,
			\begin{align}\label{eq:1}
				u(n+1) = q \cdot u(n),
			\end{align}
		où $q \in \R$ est un réel fixé qui ne dépend pas de $n$.
		
		On appelle $q$ la \emph{raison} de la suite.
	}{}