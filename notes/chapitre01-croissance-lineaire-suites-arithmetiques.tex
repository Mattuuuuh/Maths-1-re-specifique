%!TEX encoding = UTF8
%!TEX root =notes.tex

\chapter{Croissance linéaire}

	\section{Suites arithmétiques}


	\dfn{Suite}{
		Une suite $u$ est une fonction qui à tout entier naturel $n \in \N$ associe un réel
			\[ u(n) \in \R. \]
		Le \emph{terme initial} de la suite est donné par $u(0)$, et son terme de \emph{rang} $n$ est $u(n)$.
	}{}
	
	\ex{}{
		Les fonctions suivantes sont des suites données par leur rang $n$ : on connait donc leur valeur pour tout les entiers naturels.
		\begin{multicols}{2}
		\begin{enumerate}
			\item $u(n) = n+1$
			\item $v(n) = 15 + n$
			\item $\xi(n) = 3n + 1$
			\item $a(n) = n^2$
		\end{enumerate}
		\end{multicols}
		Une suite n'a pas forcément de formule générale pour tout $n$, c'est une simplement \emph{suite} de nombres réels.
	}{}
	
	%Les exercices étudiés en cours nous ont menés à considérer les suites ayant une structure particulière : les suites obtenue par ajouts successifs d'une même constante.
	
	\dfn{Suite arithmétique}{
		On dit d'une suite $u$ qu'elle est arithmétique dès qu'elle vérifie, pour tout $n\in\N$,
			\begin{align}\label{eq:1}
				u(n+1) - u(n) = r,
			\end{align}
		où $r \in \R$ est un réel fixé qui ne dépend pas de $n$.
		
		On appelle $r$ la \emph{raison} de la suite.
	}{}
	
	\nt{
		L'équation \eqref{eq:1} peut se réécrire comme
			\begin{align}\label{eq:2}
				u(n+1) = u(n) + r
			\end{align}
		Avec des mots, on lira alors \og le terme $n+1$ est égal au terme $n$ plus $r$ \fg, et ceci toujours pour tout $n\in\N$.
	}
	
	\nt{
		Pour vérifier qu'une suite est arithmétique et pour connaître sa raison, on préférera l'écriture \eqref{eq:1}.
		L'écriture \eqref{eq:2} est utile pour calculer les terms successifs d'une suite.
	}
	
	\ex{}{
		Soit $u$ la suite de terme initial $u(0) = -2$ et de raison $3$.
		
		L'équation \eqref{eq:2} spécialisée en $n=0 \in \N$ donne
			\[ u(1) = u(0) + 3 = -2 + 3 = 1. \]
		Puis, en $n=1 \in \N$, on trouve
			\[ u(2) = u(1) + 3 = 1 + 3 = 4. \]
		Etc... pour avoir
			\begin{align*}
				u(3) &= 7 & u(4) &= 10 & u(5) &= 13 & \cdots
			\end{align*}
		On conjecture que le terme de rang $n\in\N$ s'écrit alors
				\[ u(n) = -2 + 3n. \]
		Le théorème \ref{thm:lili} démontrera cette conjecture.
	}{}
	
	\ex{}{
		La suite donnée, pour tout $n\in\N$, par
			\[ u(n) = 3n - 2 \]
		est arithmétique.
		En effet, on vérifie que, pour tout $n\in \N$,
			\[ u(n+1) - u(n) = 3(n+1) - 2 - (3n - 2) = 3n + 3 - 2 - 3n + 2 = 3.\]
		Elle est donc arithmétique de raison $3$.
	}{}
	
	\nt{
		L'équation \eqref{eq:1} peut être utilisée pour déduire les variations d'une suite arithmétique selon le signe de sa raison.
		Par exemple, si $r > 0$, alors
			\[ u(n+1) > u(n) \]
		pour tout $n\in\N$
		
		On lira \og le terme $n+1$ est strictement supérieur au terme $n$, pour tout entier naturel $n$ \fg.
		De fait, $u$ est donc croissante.
	}
	
	\dfn{}{
		Pour $u$ une suite arithmétique de raison $r\in\R$. On dit que
		\[\begin{cases*} 
			& $u$ est \emph{croissante} dès que $r > 0$, \\
			& $u$ est \emph{constante} dès que $r=0$, et \\
			& $u$ est \emph{décroissante} dès que $r < 0$.
		\end{cases*}\]
	}
	
	\thm{}{
		Soit une suite arithmétique $u$ de terme initial $u(0)\in\R$ et de raison $r \in \R$.
		
		Alors, le terme de rang $n\in\N$ de $u$ est égal à
			\[ u(n) = r \cdot n + u(0). \]	
	}{thm:lili}
	
	\pf{Preuve par itération}{
		On utilise l'équation \eqref{eq:2} en la spécialisant en $n=1, 2, 3, \dots$ jusqu'au terme de rang $n$.
			\begin{align*}
				u(1) &= u(0) + r \\
				u(2) &= u(1) + r = u(0) + r + r = u(0) + 2r \\
				u(3) &= u(2) + r = u(0) + 2r + r = u(3) + 3r \\
				 &\vdots \\
				u(n) &= u(n-1) + r = \dots = u(0) + (n-1)r + r = u(0) + r \cdot n
			\end{align*}	
	}
	\pf{Preuve par somme téléscopique}{
		On utilise l'équation \eqref{eq:1} en la spécialisant de la même façon.
			\begin{align*}
				u(1) - u(0) &= r \\
				u(2) - u(1) &= r \\
				u(3) - u(2) &= r \\
				&\vdots \\
				u(n) - u(n-1) &= r
			\end{align*}
		En ajoutant chaque terme de la gauche un à un, on obtient une somme appelé \emph{téléscopique}.
		On a d'abord $u(1) - u(0)$, puis
			\[ \left(u(1) - u(0)\right) + \left(u(2) - u(1)\right) = u(2) - u(0), \]
		puis
			\[ \left(u(1) - u(0)\right) + \left(u(2) - u(1)\right) + \left(u(3) - u(2)\right) = u(3) - u(0), \]
		etc...
		Or, chaque différence est égale à $r$. La $n$-ième somme donnera donc
			\[ r + r + \dots + r = u(n) - u(0) = r \cdot n, \]
		d'où le résultat.
	}
	
	\ex{}{
		Une suite $\chi$ de terme initial $-12$ et de raison $\pi$ est donnée, pour tout $n\in\N$, par
			\[ \chi(n) = \pi n - 12. \]
	}{}
	
	\nt{
		Il est souvent fastidieux (et en fait redondant) de démontrer qu'une suite est bien arithmétique à l'aide de l'équation \eqref{eq:1} à vérifier pour tous les entiers naturels.
		
		Le théorème \ref{thm:faustine} suivant donne la réciproque du théorème \ref{thm:lili} et permet de conclure du caractère arithmétique ou non d'une suite en voyant son expression au rang $n\in\N$.
	}
	
	\thm{}{
		Soit $u$ une suite donnée par, pour tout $n\in\N$,
			\[ u(n) = r \cdot n + b. \]
		Alors $u$ est arithmétique de raison $r$ et de terme initial $u(0) = b$.
	
	}{thm:faustine}
	
	\pf{Démonstration}{
		Soit $u$ la suite de l'énoncé.
		Considérons une nouvelle suite $v$ de terme initial $b$ et de raison $r$.
		Alors, par le théorème \ref{thm:lili}, le terme de rang $n$ s'écrit, pour tout $n\in\N$,
			\[ v(n) = r \cdot n + b. \]
		On reconnaît l'expression de $u$, et donc
			\[ v(n) = u(n) \]
		pour tout $n\in\N$.
		Les suites $u$ et $v$ sont égales en tout rang et donc égales comme suites.
		Ainsi $u$ est arithmétique de raison $r$ et de terme initial $b$.
	}
	
	\nt{
		On aurait également pû démontrer le théorème \ref{thm:faustine} en utilisant la définition d'une suite arithmétique donnée en \eqref{eq:1}.
		On calcule alors calmement que, pour tout $n\in\N$,
			\begin{align*}
				u(n + 1) - u(n) &= r(n+1) + b - \left( rn + b \right) \\
							&= rn + r + b - rn -b \\
							&= r,
			\end{align*}
		et que le terme initial est donné par $u(0) = r \times 0 + b = b$
		
		Nous voilà rassurés.
	}


	\nt{
		Savoir résoudre les inégalités sur les entiers sert à répondre aux questions du type
			\begin{enumerate}
				\item À partir de quel rang la suite dépasse-t-elle une valeur donnée ?
				\item Quels sont les rangs pour lesquels une suite est inférieure (ou égale) à une autre suite ?
			\end{enumerate}
		La notion de \og dépassement \fg est à comprendre dans les deux sens : on peut cherche à savoir pour quels rangs une suite est supérieure à un certain seuil, ou inférieure.
	}
	
	\exe{}{
		Soient les suites données par, pour tout $n\in\N$,
			\[ u(n) = 3n + 1, \qquad v(n) = -3n + 12.\]
		Pour quels rangs $n\in\N$ l'inégalité suivante est-elle vérifiée ?
			\[ u(n) \leq v(n). \]
		On remarque que le terme initial de $v$ est supérieur à celui de $u$ et donc que la réponse devra au moins contenir le rang $0$.
	}{}
	
	\exe{}{
		Pour $u$ une suite donnée par, pour tout $n\in\N$,
			\[ u(n) = 3918n + 60 000. \]
		À partir de quel rang $N\in\N$ la suite $u$ dépasse-t-elle deux fois sa valeur initiale ?
	}
	\pf{Solution}{
		On cherche à trouver le premier rang $N\in\N$ tel que
			\[ u(N) \geq 2u(0). \]
		Ceci se substitut en
			\[ 3918N + 60 000 \geq 120 000. \]
		On soustrait la valeur initale des deux côtés, puis on divise par 3918.
		\textbf{Le sens de l'inégalité reste inchangé car $3918$ est positif.}
		D'où
			\[ N \geq \dfrac{60 000}{3918} \approx 15,3. \]
		Le premier $N \in \N$ vérifiant cette inégalité est donc $N=16$.
	}
	
	\exe{}{
		Pour $u$ une suite donnée par, pour tout $n\in\N$,
			\[ u(n) = -43n +450. \]
		À partir de quel rang $N\in\N$ la suite $u$ devient-elle négative ou nulle ?
	}
	
	\exe{}{
		Donner l'ensemble des $n\in\N$ vérifiants l'inégalité suivante.
			\[ -\dfrac47 n + 9 \leq 3. \]
	}
	\textbf{Attention : multiplier par un nombre strictement négatif change le sens des inégalités.}