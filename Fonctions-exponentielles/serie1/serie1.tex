% SOLUTION SWITCH
\newif\ifsolutions
				\solutionstrue
				\solutionsfalse
				
%!TEX encoding = UTF8
%!TEX root =notes.tex


%%%%%%%%%%%%%%%%%%%%%%%%%%%%%%%%%
% PACKAGE IMPORTS
%%%%%%%%%%%%%%%%%%%%%%%%%%%%%%%%%


\usepackage[french]{babel}

\usepackage[tmargin=2cm,rmargin=1in,lmargin=1in,margin=0.85in,bmargin=2cm,footskip=.2in]{geometry}
\usepackage{amsmath,amsfonts,amsthm,amssymb,mathtools}
\usepackage[varbb]{newpxmath}
\usepackage{xfrac}
\usepackage[makeroom]{cancel}
\usepackage{mathtools}
\usepackage{bookmark}
\usepackage{enumitem}
\usepackage{hyperref,theoremref}
\hypersetup{
	pdftitle={Assignment},
	colorlinks=true, linkcolor=doc!90,
	bookmarksnumbered=true,
	bookmarksopen=true
}
\usepackage[most,many,breakable]{tcolorbox}
\usepackage{xcolor}
\usepackage{varwidth}
\usepackage{varwidth}
\usepackage{etoolbox}
%\usepackage{authblk}
\usepackage{nameref}
\usepackage{multicol,array}
\usepackage{tikz-cd}
\usepackage[ruled,vlined,linesnumbered]{algorithm2e}
\usepackage{comment} % enables the use of multi-line comments (\ifx \fi) 
\usepackage{import}
\usepackage{xifthen}
\usepackage{pdfpages}
\usepackage{pgfplots}
\usepackage{transparent}


\newcommand\mycommfont[1]{\footnotesize\ttfamily\textcolor{blue}{#1}}
\SetCommentSty{mycommfont}
\newcommand{\incfig}[1]{%
    \def\svgwidth{\columnwidth}
    \import{./figures/}{#1.pdf_tex}
}

\usepackage{tikzsymbols}
%\renewcommand\qedsymbol{$\Laughey$}


%\usepackage{import}
%\usepackage{xifthen}
%\usepackage{pdfpages}
%\usepackage{transparent}


%%%%%%%%%%%%%%%%%%%%%%%%%%%%%%
% SELF MADE COLORS
%%%%%%%%%%%%%%%%%%%%%%%%%%%%%%



\definecolor{myg}{RGB}{56, 140, 70}
\definecolor{myb}{RGB}{45, 111, 177}
\definecolor{myr}{RGB}{199, 68, 64}
\definecolor{mytheorembg}{HTML}{F2F2F9}
\definecolor{mytheoremfr}{HTML}{00007B}
\definecolor{mylenmabg}{HTML}{FFFAF8}
\definecolor{mylenmafr}{HTML}{983b0f}
\definecolor{mypropbg}{HTML}{f2fbfc}
\definecolor{mypropfr}{HTML}{191971}
\definecolor{myexamplebg}{HTML}{F2FBF8}
\definecolor{myexamplefr}{HTML}{88D6D1}
\definecolor{myexampleti}{HTML}{2A7F7F}
\definecolor{mydefinitbg}{HTML}{E5E5FF}
\definecolor{mydefinitfr}{HTML}{3F3FA3}
\definecolor{notesgreen}{RGB}{0,162,0}
\definecolor{myp}{RGB}{197, 92, 212}
\definecolor{mygr}{HTML}{2C3338}
\definecolor{myred}{RGB}{127,0,0}
\definecolor{myyellow}{RGB}{169,121,69}
\definecolor{myexercisebg}{HTML}{F2FBF8}
\definecolor{myexercisefg}{HTML}{88D6D1}


%%%%%%%%%%%%%%%%%%%%%%%%%%%%
% TCOLORBOX SETUPS
%%%%%%%%%%%%%%%%%%%%%%%%%%%%

\setlength{\parindent}{1cm}
%================================
% THEOREM BOX
%================================

\tcbuselibrary{theorems,skins,hooks}
\newtcbtheorem[number within=chapter]{Theorem}{Théorème}
{%
	enhanced,
	breakable,
	colback = mytheorembg,
	frame hidden,
	boxrule = 0sp,
	borderline west = {2pt}{0pt}{mytheoremfr},
	sharp corners,
	detach title,
	before upper = \tcbtitle\par\smallskip,
	coltitle = mytheoremfr,
	fonttitle = \bfseries\sffamily,
	description font = \mdseries,
	separator sign none,
	segmentation style={solid, mytheoremfr},
}
{th}


\tcbuselibrary{theorems,skins,hooks}
\newtcolorbox{Theoremcon}
{%
	enhanced
	,breakable
	,colback = mytheorembg
	,frame hidden
	,boxrule = 0sp
	,borderline west = {2pt}{0pt}{mytheoremfr}
	,sharp corners
	,description font = \mdseries
	,separator sign none
}

%================================
% Corollery
%================================
\tcbuselibrary{theorems,skins,hooks}
\newtcbtheorem[use counter=tcb@cnt@Theorem]{Corollary}{Corollaire}
{%
	enhanced
	,breakable
	,colback = myp!10
	,frame hidden
	,boxrule = 0sp
	,borderline west = {2pt}{0pt}{myp!85!black}
	,sharp corners
	,detach title
	,before upper = \tcbtitle\par\smallskip
	,coltitle = myp!85!black
	,fonttitle = \bfseries\sffamily
	,description font = \mdseries
	,separator sign none
	,segmentation style={solid, myp!85!black}
}
{th}

%================================
% LENMA
%================================

\tcbuselibrary{theorems,skins,hooks}
\newtcbtheorem[use counter=tcb@cnt@Theorem]{Lemma}{Lemme}
{%
	enhanced,
	breakable,
	colback = mylenmabg,
	frame hidden,
	boxrule = 0sp,
	borderline west = {2pt}{0pt}{mylenmafr},
	sharp corners,
	detach title,
	before upper = \tcbtitle\par\smallskip,
	coltitle = mylenmafr,
	fonttitle = \bfseries\sffamily,
	description font = \mdseries,
	separator sign none,
	segmentation style={solid, mylenmafr},
}
{th}


%================================
% PROPOSITION
%================================

\tcbuselibrary{theorems,skins,hooks}
\newtcbtheorem[use counter=tcb@cnt@Theorem]{Prop}{Proposition}
{%
	enhanced,
	breakable,
	colback = mypropbg,
	frame hidden,
	boxrule = 0sp,
	borderline west = {2pt}{0pt}{mypropfr},
	sharp corners,
	detach title,
	before upper = \tcbtitle\par\smallskip,
	coltitle = mypropfr,
	fonttitle = \bfseries\sffamily,
	description font = \mdseries,
	separator sign none,
	segmentation style={solid, mypropfr},
}
{th}


%================================
% CLAIM
%================================

\tcbuselibrary{theorems,skins,hooks}
\newtcbtheorem[use counter=tcb@cnt@Theorem]{claim}{Claim}
{%
	enhanced
	,breakable
	,colback = myg!10
	,frame hidden
	,boxrule = 0sp
	,borderline west = {2pt}{0pt}{myg}
	,sharp corners
	,detach title
	,before upper = \tcbtitle\par\smallskip
	,coltitle = myg!85!black
	,fonttitle = \bfseries\sffamily
	,description font = \mdseries
	,separator sign none
	,segmentation style={solid, myg!85!black}
}
{th}



%================================
% Exercise
%================================

\tcbuselibrary{theorems,skins,hooks}
\newtcbtheorem[use counter=tcb@cnt@Theorem]{Exercise}{Exercice}
{%
	enhanced,
	breakable,
	colback = myexercisebg,
	frame hidden,
	boxrule = 0sp,
	borderline west = {2pt}{0pt}{myexercisefg},
	sharp corners,
	detach title,
	before upper = \tcbtitle\par\smallskip,
	coltitle = myexercisefg,
	fonttitle = \bfseries\sffamily,
	description font = \mdseries,
	separator sign none,
	segmentation style={solid, myexercisefg},
}
{th}

%================================
% EXAMPLE BOX
%================================

\newtcbtheorem[use counter=tcb@cnt@Theorem]{Example}{Exemple}
{%
	colback = myexamplebg
	,breakable
	,colframe = myexamplefr
	,coltitle = myexampleti
	,boxrule = 1pt
	,sharp corners
	,detach title
	,before upper=\tcbtitle\par\smallskip
	,fonttitle = \bfseries
	,description font = \mdseries
	,separator sign none
	,description delimiters parenthesis
}
{ex}

%================================
% DEFINITION BOX
%================================

\newtcbtheorem[use counter=tcb@cnt@Theorem]{Definition}{Définition}{enhanced,
	before skip=2mm,after skip=2mm, colback=red!5,colframe=red!80!black,boxrule=0.5mm,
	attach boxed title to top left={xshift=1cm,yshift*=1mm-\tcboxedtitleheight}, varwidth boxed title*=-3cm,
	boxed title style={frame code={
					\path[fill=tcbcolback]
					([yshift=-1mm,xshift=-1mm]frame.north west)
					arc[start angle=0,end angle=180,radius=1mm]
					([yshift=-1mm,xshift=1mm]frame.north east)
					arc[start angle=180,end angle=0,radius=1mm];
					\path[left color=tcbcolback!60!black,right color=tcbcolback!60!black,
						middle color=tcbcolback!80!black]
					([xshift=-2mm]frame.north west) -- ([xshift=2mm]frame.north east)
					[rounded corners=1mm]-- ([xshift=1mm,yshift=-1mm]frame.north east)
					-- (frame.south east) -- (frame.south west)
					-- ([xshift=-1mm,yshift=-1mm]frame.north west)
					[sharp corners]-- cycle;
				},interior engine=empty,
		},
	fonttitle=\bfseries,
	title={#2},#1}{def}

%================================
% Solution BOX
%================================

\makeatletter
\newtcbtheorem[use counter=tcb@cnt@Theorem]{question}{Question}{enhanced,
	breakable,
	colback=white,
	colframe=myb!80!black,
	attach boxed title to top left={yshift*=-\tcboxedtitleheight},
	fonttitle=\bfseries,
	title={#2},
	boxed title size=title,
	boxed title style={%
			sharp corners,
			rounded corners=northwest,
			colback=tcbcolframe,
			boxrule=0pt,
		},
	underlay boxed title={%
			\path[fill=tcbcolframe] (title.south west)--(title.south east)
			to[out=0, in=180] ([xshift=5mm]title.east)--
			(title.center-|frame.east)
			[rounded corners=\kvtcb@arc] |-
			(frame.north) -| cycle;
		},
	#1
}{def}
\makeatother

%================================
% SOLUTION BOX
%================================

\makeatletter
\newtcolorbox{solution}{enhanced,
	breakable,
	colback=white,
	colframe=myg!80!black,
	attach boxed title to top left={yshift*=-\tcboxedtitleheight},
	title=Solution,
	boxed title size=title,
	boxed title style={%
			sharp corners,
			rounded corners=northwest,
			colback=tcbcolframe,
			boxrule=0pt,
		},
	underlay boxed title={%
			\path[fill=tcbcolframe] (title.south west)--(title.south east)
			to[out=0, in=180] ([xshift=5mm]title.east)--
			(title.center-|frame.east)
			[rounded corners=\kvtcb@arc] |-
			(frame.north) -| cycle;
		},
}
\makeatother

%================================
% Question BOX
%================================

\makeatletter
\newtcbtheorem[use counter=tcb@cnt@Theorem]{qstion}{Question}{enhanced,
	breakable,
	colback=white,
	colframe=mygr,
	attach boxed title to top left={yshift*=-\tcboxedtitleheight},
	fonttitle=\bfseries,
	title={#2},
	boxed title size=title,
	boxed title style={%
			sharp corners,
			rounded corners=northwest,
			colback=tcbcolframe,
			boxrule=0pt,
		},
	underlay boxed title={%
			\path[fill=tcbcolframe] (title.south west)--(title.south east)
			to[out=0, in=180] ([xshift=5mm]title.east)--
			(title.center-|frame.east)
			[rounded corners=\kvtcb@arc] |-
			(frame.north) -| cycle;
		},
	#1
}{def}
\makeatother

\newtcbtheorem[number within=chapter]{wconc}{Wrong Concept}{
	breakable,
	enhanced,
	colback=white,
	colframe=myr,
	arc=0pt,
	outer arc=0pt,
	fonttitle=\bfseries\sffamily\large,
	colbacktitle=myr,
	attach boxed title to top left={},
	boxed title style={
			enhanced,
			skin=enhancedfirst jigsaw,
			arc=3pt,
			bottom=0pt,
			interior style={fill=myr}
		},
	#1
}{def}



%================================
% NOTE BOX
%================================

\usetikzlibrary{arrows,calc,shadows.blur}
\tcbuselibrary{skins}
\newtcolorbox{note}[1][]{%
	enhanced jigsaw,
	colback=gray!20!white,%
	colframe=gray!80!black,
	size=small,
	boxrule=1pt,
	title=\textbf{Remarque},
	halign title=flush center,
	coltitle=black,
	breakable,
	drop shadow=black!50!white,
	attach boxed title to top left={xshift=1cm,yshift=-\tcboxedtitleheight/2,yshifttext=-\tcboxedtitleheight/2},
	minipage boxed title=2.6cm,
	boxed title style={%
			colback=white,
			size=fbox,
			boxrule=1pt,
			boxsep=2pt,
			underlay={%
					\coordinate (dotA) at ($(interior.west) + (-0.5pt,0)$);
					\coordinate (dotB) at ($(interior.east) + (0.5pt,0)$);
					\begin{scope}
						\clip (interior.north west) rectangle ([xshift=3ex]interior.east);
						\filldraw [white, blur shadow={shadow opacity=60, shadow yshift=-.75ex}, rounded corners=2pt] (interior.north west) rectangle (interior.south east);
					\end{scope}
					\begin{scope}[gray!80!black]
						\fill (dotA) circle (2pt);
						\fill (dotB) circle (2pt);
					\end{scope}
				},
		},
	#1,
}

%================================
% STRATÉGIE BOX
%================================

\usetikzlibrary{arrows,calc,shadows.blur}
\tcbuselibrary{skins}
\newtcolorbox{strategy}[1][]{%
	enhanced jigsaw,
	colback=myb!20!white,%
	colframe=gray!80!black,
	size=small,
	boxrule=1pt,
	title=\textbf{Stratégie},
	halign title=flush center,
	coltitle=black,
	breakable,
	drop shadow=black!50!white,
	attach boxed title to top left={xshift=1cm,yshift=-\tcboxedtitleheight/2,yshifttext=-\tcboxedtitleheight/2},
	minipage boxed title=2.5cm,
	boxed title style={%
			colback=white,
			size=fbox,
			boxrule=1pt,
			boxsep=2pt,
			underlay={%
					\coordinate (dotA) at ($(interior.west) + (-0.5pt,0)$);
					\coordinate (dotB) at ($(interior.east) + (0.5pt,0)$);
					\begin{scope}
						\clip (interior.north west) rectangle ([xshift=3ex]interior.east);
						\filldraw [white, blur shadow={shadow opacity=60, shadow yshift=-.75ex}, rounded corners=2pt] (interior.north west) rectangle (interior.south east);
					\end{scope}
					\begin{scope}[gray!80!black]
						\fill (dotA) circle (2pt);
						\fill (dotB) circle (2pt);
					\end{scope}
				},
		},
	#1,
}

%%%%%%%%%%%%%%%%%%%%%%%%%%%%%%%%%%%%%%%%%%%
% TABLE OF CONTENTS
%%%%%%%%%%%%%%%%%%%%%%%%%%%%%%%%%%%%%%%%%%%

\usepackage{tikz}

\definecolor{doc}{RGB}{0,60,110}
\usepackage{titletoc}
\contentsmargin{0cm}
\titlecontents{chapter}[3.7pc]
{\addvspace{30pt}%
	\begin{tikzpicture}[remember picture, overlay]%
		\draw[fill=doc!60,draw=doc!60] (-7,-.1) rectangle (-0.2,.6);%
		\pgftext[left,x=-3.5cm,y=0.2cm]{\color{white}\Large\sc\bfseries Chapitre\ \thecontentslabel};%
	\end{tikzpicture}\color{doc!60}\large\sc\bfseries}%
{}
{}
{\;\titlerule\;\large\sc\bfseries Page \thecontentspage
	\begin{tikzpicture}[remember picture, overlay]
		\draw[fill=doc!60,draw=doc!60] (2pt,0) rectangle (4,0.1pt);
	\end{tikzpicture}}%
\titlecontents{section}[3.7pc]
{\addvspace{2pt}}
{\contentslabel[\thecontentslabel]{2pc}}
{}
{\hfill\small \thecontentspage}
[]
\titlecontents*{subsection}[3.7pc]
{\addvspace{-1pt}\small}
{}
{}
{\ --- \small\thecontentspage}
[ \textbullet\ ][]

\makeatletter
\renewcommand{\tableofcontents}{%
	\chapter*{%
	  \vspace*{-20\p@}%
	  \begin{tikzpicture}[remember picture, overlay]%
		  \pgftext[right,x=15cm,y=0.2cm]{\color{doc!60}\Huge\sc\bfseries \contentsname};%
		  \draw[fill=doc!60,draw=doc!60] (13,-.75) rectangle (20,1);%
		  \clip (13,-.75) rectangle (20,1);
		  \pgftext[right,x=15cm,y=0.2cm]{\color{white}\Huge\sc\bfseries \contentsname};%
	  \end{tikzpicture}}%
	\@starttoc{toc}}
\makeatother


%%%%%%%%%%%%%%%%%%%%%%%%%%%%%%%%%%%%%%%%%%%
% MINTED FOR PYTHON ALGORITHMS
%%%%%%%%%%%%%%%%%%%%%%%%%%%%%%%%%%%%%%%%%%%

\usepackage{tcolorbox}
\tcbuselibrary{minted,breakable,xparse,skins}
\definecolor{bg}{gray}{0.95}
\DeclareTCBListing{mintedbox}{O{}m!O{}}{%
  breakable=true,
  listing engine=minted,
  listing only,
  minted language=#2,
  minted style=default,
  minted options={%
    linenos,
    gobble=0,
    breaklines=true,
    breakafter=,,
    fontsize=\small,
    numbersep=8pt,
    #1},
  boxsep=0pt,
  left skip=0pt,
  right skip=0pt,
  left=25pt,
  right=0pt,
  top=3pt,
  bottom=3pt,
  arc=5pt,
  leftrule=0pt,
  rightrule=0pt,
  bottomrule=2pt,
  toprule=2pt,
  colback=bg,
  colframe=orange!70,
  enhanced,
  overlay={%
    \begin{tcbclipinterior}
    \fill[orange!20!white] (frame.south west) rectangle ([xshift=20pt]frame.north west);
    \end{tcbclipinterior}},
  #3}
  
  
 % for braces
\usetikzlibrary{decorations.pathreplacing}



\AdvanceDate[1]

\begin{document}
\pagestyle{fancy}
\fancyhead[L]{Première}
\fancyhead[C]{\textbf{Croissance exponentielle : évolution relative \ifsolutions -- Solutions \fi}}
\fancyhead[R]{\today}


\begin{definition*}{Pourcentage}{}
	Soient $P, x \in\R$ deux nombres réels positifs ou nuls.
	Alors
	  	\begin{center}
	  	\og $P \%$ de $x$ \fg~= \dots\dots\dots\dots\dots
	  	\end{center}
	En particulier,
	  	\begin{center}
	  	\og $A \%$ de $B$ \fg~= \og $B \%$ de $A$ \fg.
	  	\end{center}
\end{definition*}

\begin{theorem}[label=thm:ev-succ]{Évolutions successives}{}
	L'évolution d'une valeur correspond à sa multiplication par un coefficient multiplicateur $m$.
	
	\begin{enumerate}
		\item Si $m>1$, $m=1+p$, et l'évolution est une augmentation de $100p \%$.
		\item Si $m<1$, $m=1-p$, et l'évolution est une diminution de $100p \%$.
		\item Si $m=1$, la valeur ne change pas.
	\end{enumerate}
	
	Lorsque deux évolutions successives ont lieu, les coefficients sont multipliés entre eux pour obtenir un coefficient multiplicateur global.

	\begin{center}
	\begin{tikzpicture}
		% nodes
		\draw (0,0) ellipse (2cm and .5cm) node {Valeur initiale};
		
		\draw (5,0) ellipse (2cm and .5cm) node {Nouvelle valeur};
		
		\draw (10,0) ellipse (2cm and .5cm) node {Valeur finale};
		
		% vertices
		\draw[->, thick, myg] (1cm,.6cm) arc (105:75:7) node[midway, above] {$\times m_1$};
		\draw[->, thick, myg] (6cm,.6cm) arc (105:75:7) node[midway, above] {$\times m_2$};
		
		\draw[->, thick, myr] (1cm,-.5cm) arc (-105:-75:15) node[midway, below=5pt] {$\times \dots\dots\dots\dots$};
	\end{tikzpicture}
	\end{center}
\end{theorem}

\begin{theorem}{Évolution réciproque}{}
	L'évolution réciproque est l'évolution qui permet de revenir à une valeur initale.
	
	\begin{center}
	\begin{tikzpicture}
		% nodes
		\draw (0,0) ellipse (2cm and .5cm) node {Valeur initiale};
		
		\draw (5,0) ellipse (2cm and .5cm) node {Nouvelle valeur};
		
		\draw (10,0) ellipse (2cm and .5cm) node {Valeur initiale};
		
		% vertices
		\draw[->, thick, myg] (1cm,.6cm) arc (105:75:7) node[midway, above] {$\times m_1$};
		\draw[->, thick, myg] (6cm,.6cm) arc (105:75:7) node[midway, above] {$\times \dots\dots$};
		
		\draw[->, thick, myr] (1cm,-.5cm) arc (-105:-75:15) node[midway, below=5pt] {$\times \dots\dots$};
	\end{tikzpicture}
	\end{center}
\end{theorem}

\begin{theorem}{Évolution moyenne}{}
	Si une valeur de départ subit $n$ évolutions successives, alors on appelle \emph{coefficient multiplicateur moyen} le nombre
		%\[ M_{\text{Moyen}} = \left( M_{\text{Global}} \right)^{1/n}. \]
		\[ M_{\text{Moyen}} = \dots\dots\dots\dots\dots\dots\dots \]

	\begin{center}
	\begin{tikzpicture}[scale=.8]
		% nodes
		\draw (0,0) ellipse (2cm and .5cm) node {Valeur $1$};
		
		\draw (5,0) ellipse (2cm and .5cm) node {Valeur $2$};
		
		\draw (10,0) ellipse (2cm and .5cm) node {Valeur $3$};
		
		\draw (15,0) ellipse (2cm and .5cm) node {Valeur $4$};
		
		% vertices
		\draw[->, thick, myg] (1cm,.6cm) arc (105:75:7) node[midway, above] {$\times m_1$};
		\draw[->, thick, myg] (6cm,.6cm) arc (105:75:7) node[midway, above] {$\times m_2$};
		\draw[->, thick, myg] (11cm,.6cm) arc (105:75:7) node[midway, above] {$\times m_3$};
		
		\draw[->, thick, myr] (1cm,-.5cm) arc (-105:-75:25) node[midway, below] {$\times M_{\text{Global}}$};
	\end{tikzpicture}
	\end{center}
\end{theorem}

\newpage

\exe{
  Calculer sans calculatrice les valeurs suivantes.
  \begin{multicols}{2}
    \begin{enumerate}
    \item $75\%$ de $60$
    \item $60\%$ de $75$
    \item $72\%$ de $25$
    \item $68\%$ de $20$
    \item $125\%$ de $40$
    \item $40\%$ de $125$
    \end{enumerate}
  \end{multicols}
}{
  \begin{multicols}{2}
    \begin{enumerate}
    \item $\dfrac34 \cdot 60 = 3 \cdot \dfrac{60}4 = 3 \cdot 15 = 45$
    \item $45$
    \item $\dfrac14 \cdot 72 = 18$
    \item $\dfrac15 \cdot 68 = \dfrac{136}{10} = 13,6$
    \item $40 + \dfrac14 \cdot 40 = 50$
    \item $50$
    \end{enumerate}
  \end{multicols}
}

\exe{
  Approximer sans calculatrice les valeurs suivantes.
  \begin{multicols}{2}
    \begin{enumerate}
    \item $33\%$ de $150$
    \item $166\%$ de $180$
    \item $11\%$ de $90$
    \item $89\%$ de $81$
    \item $16,6\%$ de $18$
    \item $83,4\%$ de $36$
    \end{enumerate}
  \end{multicols}
}{
  \begin{multicols}{2}
    \begin{enumerate}
    \item $\approx \dfrac13 \cdot 150 = 50$
    \item $\approx 180 + \dfrac23 \cdot 180 = 180 + 120 = 300$
    \item $\approx \dfrac19 \cdot 90 = 10$
    \item $\approx 81 - \dfrac19 \cdot 81 = 81 - 9 = 72$
    \item $\approx \dfrac16 \cdot 18 = 3$
    \item $\approx 36 - \dfrac16 \cdot 36 = 36 - 6 = 30$
    \end{enumerate}
  \end{multicols}
}

\exe{
  On estime la biomasse totale des fourmis sur Terre à $12$ millions de tonnes.
  Ceci serait égal à $20\%$ de la biomasse humaine.

  Estimer la biomasse totale des humains sur Terre en tonnes.
}{
	On a la relation
		\[ \dfrac{\text{biomasse des fourmis}}{\text{biomasse humaine}} = 0,2. \]
	D'où
		\[ \text{biomasse humaine} = \dfrac{12 \times 10^6}{0,2} \text{T} = 60 \times 10^6 \text{T}.\] 

}

\exe{
  Considérons deux tailleurs, l'un à $250$€ et l'autre à $360$€.
  \begin{enumerate}
  \item Quelle augmentation de prix faut-il appliquer au premier tailleur pour qu'il ait le prix du second ?
  \item Quel rabais faut-il appliquer au second tailleur pour qu'il ait le prix du premier ?
  \end{enumerate}
}{
  \begin{enumerate}
  \item On calcule l'évolution $\dfrac{360}{250}  = 1,44 = 144\%$. Ainsi, le deuxième tailleur vaut $144\%$ du prix du premier : une augmentation de $44\%$ est nécessaire.
  \item On calcule l'évolution $\dfrac{250}{360}  \approx 0,7 = 70\%$. Le premier tailleur vaut environ $70\%$ du prix du second : une diminution de $30\%$ est nécessaire.
  \end{enumerate}
}

\exe{
        À quelle évolution correspond une augmentation de $20\%$ suivie d'une diminution de $20\%$ ?
}{
	Augmenter une quantité $N$ de $20\%$ correspond à la multiplier par $1,2$.
	Une diminution, elle, multiplie par $0,8$.
	
	La quantité finale est donné par 
		\[ 0,8 \cdot (1,2 \cdot N) = (0,8 \cdot 1,2) \cdot N = 0,96 \cdot N, \]
	qui correspond à une diminution de $4\%$.
}

\exe{
  Si on augmente le prix d'un objet de $150\%$, quel rabais faut-il appliquer pour retrouver le prix initial de l'objet ?
}{
	Notons $P$ le prix initial de l'objet.
	Le prix augmenté vaut donc $1,5 \cdot P$.
	Pour retrouver $P$, il faut multiplier le prix augmenté par l'inverse de $1,5$, soit $1,5^{-1} = \dfrac23 \approx 0,666 = 66,6\%$.
	Ceci correspond à une diminution de $33,4\%$.
}

\exe{
	
	Le nombre de visiteurs du Musée du Louvre au cours des années $2015$ à $2018$ sont donnés ci-dessous.
	\begin{center}
	\begin{tikzpicture}[scale=.8]
		% nodes
		\draw (0,0) ellipse (2cm and .5cm) node {$8,6$ millions};
		
		\draw (5,0) ellipse (2cm and .5cm) node {$7,3$ millions};
		
		\draw (10,0) ellipse (2cm and .5cm) node {$8,1$ millions};
		
		\draw (15,0) ellipse (2cm and .5cm) node {$10,2$ millions};
		
		% vertices
		\draw[->, thick, myg] (1cm,.6cm) arc (105:75:7);
		\draw[->, thick, myg] (6cm,.6cm) arc (105:75:7);
		\draw[->, thick, myg] (11cm,.6cm) arc (105:75:7);
		
		\draw[->, thick, myr] (1cm,-.5cm) arc (-105:-75:25);
	\end{tikzpicture}
	\end{center}

	\begin{enumerate}
		\item Compléter le schéma en ajoutant les coefficients multiplicateurs.
		\item Calculer le coefficient multiplicateur moyen et en déduire le taux d'évolution moyen.
	\end{enumerate}

}{

}

\exe{
	
	Le nombre de visiteurs du Musée du Louvre semble augmenter de $6\%$ chaque année.
	À l'année $0$, on compte $8,6$ millions de visiteurs.

	\begin{enumerate}
		\item Écrire $V(n)$, le nombre de visiteurs du musée après $n$ années, où $n\in\N$ est un entier naturel.
		\item Écrire $V(x)$, le nombre de visiteurs du musée après $x$ années, où $x\geq0$ est un nombre réel positif ou nul.
		\item À partir de quand le nombre de visiteurs dépassera $13,3$ millions ?
		\item À partir de quand le nombre de visiteurs dépassera $14,78$ millions ?
	\end{enumerate}

}{

}


%\newpage
%
%\subsection*{Exercices supplémentaires}
%
%
%\exe{
%  En $2023$, le prix moyen du gaz naturel facturé aux ménages français s'élève à $115$€ par MWh, toutes taxes comprises (TTC).
%  En $2022$, le prix était de $96$€.
%
%  Calculer le pourcentage d'augmentation du prix entre l'année $2022$ et l'année $2023$.
%}{
%	On calcule la proportion $\dfrac{115}{96} \approx 1,20 = 120\%$.
%	Celle-ci correspond à une augmentation de $20\%$.
%}
%
%\exe{
%  En $2022$ en France, la consommation de gaz naturel s'établit à $463$ TWh.
%  En $2021$, celle-ci s'élevait plutôt à $475{,}85$ TWh.
%
%  Calculer le pourcentage de diminution de la consommation entre l'année $2021$ et l'année $2022$.
%}{
%	On calcule la proportion $\dfrac{463}{475{,}85} \approx 0,973 = 97,3\%$.
%	Celle-ci correspond à une diminution de $2,7\%$.
%}
%
%\exe{
%  Si on augmente le prix d'un objet de $100p\%$ ($p\geq0$ réel), quel rabais faut-il appliquer (en fonction de $p$) pour retrouver le prix initial de l'objet ?
%}{
%	Soit $N\geq0$ un prix quelconque, et $(1+p)\cdot N$ le prix augmenté de $100p\%$.
%	
%	Pour retrouver le prix original, il faut multiplier par $(1+p)^{-1} = \dfrac{1}{1+p}$.
%	La diminution correspondante est donnée par
%		\[ 1 - \dfrac{1}{1+p} = \dfrac{p}{1+p}. \]
%	On comparera avec l'exercice 10, où $p=0,5$, et $\dfrac{p}{1+p} = \dfrac{0,5}{1,5} = \dfrac13 \approx 33,3 \%$.
%}
%
%\exe{
%    Considérons $p\geq 0$ une proportion et $100p$ le pourcentage associé.
%    \begin{enumerate}
%    \item À quelle évolution, en fonction de $p$, correspond une augmentation de $100p\%$ suivie d'une diminution de $100p\%$ ?
%    \item Quel $p$ choisir pour trouver une diminution finale de $16\%$ ?
%    \end{enumerate}
%}{
%
%    \begin{enumerate}
%    \item Soit $N\geq0$ une quantité. Après une augmentation de $100p\%$ puis une diminution de $100p\%$,
%    		la quantité est donnée par $(1-p) \cdot (1+p) \cdot N = (1-p^2) \cdot N$.
%    		Ceci correspond à une diminution de $100\left(p^2\right) \%$.
%    		
%    		On comparera avec l'exercice 9, où $p=0,2$ et $p^2 = 0,04 = 4\%$.
%    \item On pose l'égalité suivante
%    		\[ 100p^2 = 16. \]
%    	Remarquons que $16$ et $100$ sont tous les deux des carrés parfaits :
%    		\[ p^2 = \dfrac{16}{100} = \left( \dfrac{4}{10} \right)^2, \]
%	et donc $p = \dfrac{4}{10} = 40\%$, car $p\geq 0$.
%	
%	Vérification : au vu de la question 1, on calcule $0,4^2 = 0,16 = 16\%$.
%    \end{enumerate}
%
%}
\end{document}