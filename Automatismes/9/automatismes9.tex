\documentclass[14pt]{beamer}
\usepackage[french]{babel}


\usetheme{CambridgeUS}
\usecolortheme{rose}
\beamertemplatenavigationsymbolsempty

\usepackage{array}
\usepackage{amsmath,amsfonts,amsthm,calrsfs,mathtools}
\newcolumntype{P}[1]{>{\centering\arraybackslash}p{#1}}


\usepackage{stackengine}
\newcommand\xrowht[2][0]{\addstackgap[.5\dimexpr#2\relax]{\vphantom{#1}}}


% corps
\usepackage{calrsfs}
\newcommand{\C}{\mathcal{C}}
\newcommand{\R}{\mathbb{R}}
\newcommand{\Rnn}{\mathbb{R}^{2n}}
\newcommand{\Z}{\mathbb{Z}}
\newcommand{\N}{\mathbb{N}}
\newcommand{\Q}{\mathbb{Q}}

% domain
\newcommand{\D}{\mathbb{D}}


% date
\usepackage{advdate}
\AdvanceDate[1]


\usepackage{pgfplots, subcaption}
\definecolor{myg}{RGB}{56, 140, 70}
\definecolor{myb}{RGB}{45, 111, 177}
\definecolor{myr}{RGB}{199, 68, 64}

\begin{document}

\section{Automatismes n°9}

\begin{frame}

\centering \huge
Automatismes

\end{frame}

\subsection{Suites géométriques}

\begin{frame}{1}
    Soit $P$ une suite géométrique de raison $7$ et de terme initial
        \[ P(0) = 3. \]
    Donner $P(1)$.
\end{frame}

\begin{frame}{2}
    Soit $f$ une suite géométrique de raison $11$ vérifiant
        \[ f(12) = 33. \]
    Donner $f(11)$.
\end{frame}

\begin{frame}{3}
	Donner la raison $q$ et le terme initial $v(0)$ de la suite $v$ donnée algébriquement pour tout $n\in\N$ par 
		\[ v(n) = 3 \times \left( \dfrac47 \right)^n. \]
\end{frame}

\begin{frame}{4}
	Donner le terme initial $S(0)$ de la suite $S$ donnée algébriquement pour tout $n\in\N$ par 
		\[ S(n) = 7 \times \left( \dfrac47 \right)^{3n+1}. \]
\end{frame}


\end{document}