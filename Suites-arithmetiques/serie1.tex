\documentclass[12pt]{paper}
\usepackage[french]{babel}
\usepackage[
a4paper,
margin=2cm,
nomarginpar,% We don't want any margin paragraphs
]{geometry}
\usepackage{fancyhdr}
\usepackage{array}
\usepackage{amsmath,amsfonts,amsthm,amssymb,mathtools,}
\newcolumntype{P}[1]{>{\centering\arraybackslash}p{#1}}


\usepackage{stackengine}
\newcommand\xrowht[2][0]{\addstackgap[.5\dimexpr#2\relax]{\vphantom{#1}}}

% theorems

\theoremstyle{plain}
\newtheorem{theorem}{Th\'eor\`eme}
\newtheorem{Sol}{Solution}
\newtheorem*{Sol*}{Solution}
\theoremstyle{definition}
\newtheorem{ex}{Exercice}


% corps
\newcommand{\C}{\mathbb{C}}
\newcommand{\R}{\mathbb{R}}
\newcommand{\Rnn}{\mathbb{R}^{2n}}
\newcommand{\Z}{\mathbb{Z}}
\newcommand{\N}{\mathbb{N}}
\newcommand{\Q}{\mathbb{Q}}

% domain
\newcommand{\D}{\mathbb{D}}


% date
\usepackage{advdate}
\AdvanceDate[1]

% plots
\usepackage{pgfplots}

% for calligraphic C
\usepackage{calrsfs}

% euro
\usepackage{lmodern,textcomp}

\begin{document}
\pagestyle{fancy}
\fancyhead[L]{Première G5}
\fancyhead[C]{\textbf{Suites arithmétiques 1}}
\fancyhead[R]{\today}


\begin{ex}[Prêt à taux d'intérêt fixe]
	Un étudiant souhaite faire un prêt de $60 \ 000$€ 	 à taux d'intérêt fixe à une banque.
	
	La banque lui propose un prêt à long terme, maximum $25$ ans, avec un taux d'intérêt annuel de $6{,}53 \%$.
	À la fin de chaque année, l'étudiant doit alors payer à la banque $6{,}53 \%$ de la somme initiale en plus.
	
	\begin{enumerate}
		\item Combien d'intérêts a-t-il cumulé au total à la fin de la $n$-ième année ?
		
		\item Donner la somme totale à reverser à la banque à la fin de la $n$-ième année pour rembourser le prêt.
		
		\item Après $25$ ans, combien d'argent l'étudiant doit-il rembourser au total ?
		
		\item À partir de combien d'années la valeur à rembourser est-elle supérieure au double de la valeur initiale empruntée ?
		
		\item Est-il plus avantageux de prendre, pour la même somme d'argent empruntée, un taux d'intérêt de $6{,}53\%$ sur $15$ ans ou $10\%$ sur $10$ ans ?
		
	\end{enumerate}
	
	\emph{Données : $6 \times 653 = 3918$, \qquad $3918 \times 15 = 58 \ 770.$}
\end{ex}


\begin{ex}[Raison et terme initial]

	Écrire le terme de rang $n \in \N$ des suites suivantes.
	
	\begin{enumerate}
		\item La suite $u$ de raison $-3$ et de terme initial $2$.
		\item La suite $v$ de raison $2\pi$ et de terme initial $-\frac23$.
		\item La suite $w$ de raison $0$ et de terme initial $-1$.
	\end{enumerate}
	
\end{ex}

\begin{ex}

Nommer et donner le terme de rang $n \in \N$ des suites $\star$, $\bullet$, et $\square$ suivantes données graphiquement.

\begin{center}
\begin{tikzpicture}[>=stealth, scale=1.1]
	\begin{axis}[xmin = 0, xmax=4.9, xtick={ 0,1,2, 3, 4,5}, ymin=-2, ymax=5, ytick={-1,0,1,2, 3,4}, axis x line=middle, axis y line=middle, axis line style=->, xlabel={$n$}, ylabel={}, grid=both]
		\addplot[black, thick, only marks, mark=*] coordinates {(0,1) (1,1.5) (2,2) (3,2.5) (4,3)};
		
		\addplot[black, thick, only marks, mark=star] coordinates {(1, 4.5) (2,2.5) (3,0.5) (4,-1.5)};
		
		\addplot[black, thick, only marks, mark=square] coordinates {(0,4) (1,4) (2,4) (3,4) (4,4)};
	\end{axis}

\end{tikzpicture}
\end{center}

\end{ex}

\begin{ex}
	Soient $u, v$, et $w$ trois suites arithmétiques données par
		\[ u(n ) = 3n + 4, \qquad v(n) = 2n -10, \qquad w(n) = 34 - n, \qquad \text{ pour } n \in \N.\]
	\begin{enumerate}
		\item Donner les variations de $u$ et de $w$ ainsi que leur terme initial.
		\item À partir de quel rang la suite $u$ est-elle plus grande ou égale à la suite $w$ ?
		\item À partir de quel rang la suite $w$ est-elle négative ou nulle ?
		\item La suite $v$ peut-elle être plus grande ou égale à la suite $u$ pour certains rangs $n\in\N$ ?
	\end{enumerate}
\end{ex}


\end{document}