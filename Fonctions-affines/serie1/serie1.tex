\documentclass[12pt]{paper}
\usepackage[french]{babel}
\usepackage[
a4paper,
margin=2cm,
nomarginpar,% We don't want any margin paragraphs
]{geometry}
\usepackage{fancyhdr}
\usepackage{array}
\usepackage{amsmath,amsfonts,amsthm,amssymb,mathtools,}
\newcolumntype{P}[1]{>{\centering\arraybackslash}p{#1}}


\usepackage{stackengine}
\newcommand\xrowht[2][0]{\addstackgap[.5\dimexpr#2\relax]{\vphantom{#1}}}

% theorems

\theoremstyle{plain}
\newtheorem{theorem}{Th\'eor\`eme}
\newtheorem*{sol}{Solution}
\theoremstyle{definition}
\newtheorem{ex}{Exercice}


% corps
\newcommand{\C}{\mathbb{C}}
\newcommand{\R}{\mathbb{R}}
\newcommand{\Rnn}{\mathbb{R}^{2n}}
\newcommand{\Z}{\mathbb{Z}}
\newcommand{\N}{\mathbb{N}}
\newcommand{\Q}{\mathbb{Q}}

% domain
\newcommand{\D}{\mathbb{D}}


% date
\usepackage{advdate}
\AdvanceDate[0]

% plots
\usepackage{tikz}
\usepackage{multicol}

% for calligraphic C
\usepackage{calrsfs}

% euro
\usepackage{lmodern,textcomp}


% SOLUTION SWITCH
\newif\ifsolutions
				\solutionstrue
				%\solutionsfalse

\ifsolutions
	\newcommand{\exe}[2]{
		\begin{ex} #1  \end{ex}
		\begin{sol} #2 \end{sol}
	}
\else
	\newcommand{\exe}[2]{
		\begin{ex} #1  \end{ex}
	}
	
\fi

\begin{document}
\pagestyle{fancy}
\fancyhead[L]{Première G2}
\fancyhead[C]{\textbf{Fonctions affines \ifsolutions -- Solutions \fi}}
\fancyhead[R]{\today}

\exe{
	Un vase droit, de base carrée de $5$cm de côté, a une hauteur de $20$cm.
	On y dépose une couche initial de sable de $2$cm
	
	\begin{enumerate}
		\item Quel est le volume du vase ?
		\item Quel est le volume de la couche initiale de sable ?
	\end{enumerate}
	
	On note $x$ la hauteur supplémentaire de sable que l'on rajoute.
	\begin{enumerate}
		\item À quel intervalle appartient $x$ ?
		\item Exprimer, en fonction de $x$, le volume total $V(x)$ de sable dans le vase.
		\item Calculer $V(0), V(3),$ et $V(18)$.
		\item Quelle hauteur de sable a été rajoutée si le volume total est de $335 \text{cm}^3$ ?
		\item Quelle hauteur minimale de sable a été rajoutée si le volume total dépasse $440 \text{cm}^3$ ?
	\end{enumerate}
}{
	\begin{enumerate}
		\item On fait base $\times$ hauteur, d'où $5^2 \cdot 20 = 500 cm^3$.
		\item Idem, la hauteur maintenant étant de $2$cm : $5^2 \cdot 2 = 50cm^3$.
	\end{enumerate}
	
	
	\begin{enumerate}
		\item La valeur minimale de $x$ est $0$ (hauteur non nulle), et sa valeur maximale est $18$, car il y a déjà une couche de $2$cm présente.
		
		$x$ peut donc prendre toutes les valeurs entre $0$ et $18$ incluses, d'où $x \in [0;18]$.
		\item La base reste d'aire $5^2 = 25cm^2$, et la hauteur du sable est de $2+x$cm, car il y a déjà une couche de $2$cm présente.
		
		D'où $V(x) = 25\cdot(2+x) = 25x + 50$.
		$V$ est affine.
		\item $V(0) = 50$ et $V(18) = 500$, ce qui est cohérent avec les valeurs trouvées aux deux premières questions de l'exercice.
		
			$V(3) = 25\cdot3 + 50 = 125$.
		\item On résoud pour $x$ l'équation $V(x) = 335$.
			\begin{align*}
				V(x) &= 335 \\
				25x + 50 &= 335 \\
				25x &= 285 \\
				x &= \dfrac{285}{25} = \dfrac{570}{50} = \dfrac{1140}{100} = 11,4.
			\end{align*}
		\item On résoud pour $x$ l'inéquation $V(x) \geq 440$.
			\begin{align*}
				V(x) &\geq 440 \\
				25x + 50 &\geq 440 \\
				25x &\geq 390 \\
				x &\geq 15,6
			\end{align*}
			La hauteur minimale ajoutée est donc $15,6$cm, qui est bien dans l'intervalle de définition de $x$.
	\end{enumerate}

}
\newpage
\exe{ \hspace{1cm} \\
	\begin{multicols}{2}
		\begin{center}
		\begin{tikzpicture}[scale=0.8]
		% real line
		\draw[black, thick] (0,0) -- (7,0);
		\draw[black,thick] (7,0) -- (7,7);
		\draw[black,thick] (7,7) -- (2,7);
		\draw[black,thick] (2,7)--(2,2);
		\draw[black, thick](2,2)--(0,2);
		\draw[black,thick](0,2)--(0,0);
		
		\draw[black,thick, dotted](2,0)--(2,2);
		
		\draw[<->, thick] (2,-.2) -- (7,-.2) node [midway, below] {$5$} ;
		\draw[<->, thick] (0,-.2) -- (2,-.2) node [midway, below] {$x$} ;
		\draw[<->, thick] (-.2,0) -- (-.2,2) node [midway, left] {$x$} ;
		\draw[<->, thick] (7.2,0) -- (7.2,7) node [midway, right] {$12$} ;
	\end{tikzpicture}
	\end{center}
	
	Considérons la figure ci-contre. La longueur du côté du carré de gauche doit rester inférieure à la longueur du rectangle de droite.
	\begin{enumerate}
		\item À quel intervalle appartient $x$ ?
		\item Exprimer le périmètre de la figure en fonction de $x$. Est-ce une fonction affine ? Si oui, donner son coefficient directeur de son ordonnée à l'origine.
		\item Donner l'ensemble des valeurs de $x$ pour lesquelles le périmètre est supérieur ou égal à $50$.
	\end{enumerate}
	
	\end{multicols}
}
{
	
	\begin{enumerate}
		\item En considérant les valeurs minimales et maximales de $x$, on trouve $x \in [0;12]$.
		\item Le périmètre est la longueur du pourtour, soit $12 + 5 + 5 + 12 + 2x = 34 + 2x$.
		
		La fonction $f(x) = 2x+34$ est affine, de coefficient directeur $2$ et d'ordonnée à l'origine $34$.
		\item On résoud pour $x$ l'inéquation $f(x) \geq 50$.
			\begin{align*}
				f(x) &\geq 50 \\
				34+2x &\geq 50 \\
				2x &\geq 16 \\
				x &\geq 8
			\end{align*}
		L'ensemble des valeurs est donc donné par $[8;12]$, car on a toujours $x \leq 12$ d'après la question 1.
	\end{enumerate}

}

\exe{
	Soit $(d)$ un droite, courbe de la fonction
		\[ f(x) = ax + b, \]
	et soient $A(x_A,y_A)$ et $B(x_B,y_B)$ deux points distincts appartenant à la droite.
	
	Montrer que 
		\[ a  = \dfrac{y_B - y_A}{x_B-x_A}. \]
}{

	L'appartenance des points à la droite donne les équations suivantes.
		\[ \begin{cases*}
			y_A = ax_A + b, \\
			y_B = ax_B + b.
		\end{cases*} \]
	En soustrayant la première équation à la deuxième, on trouve
		\[ y_B -y_A = (ax_B + b)-(ax_A + b) = a(x_B - x_A). \]
	D'où $x = \dfrac{y_B - y_A}{x_B-x_A}$.

}

\exe{
	Soit $(d)$ une droite du plan donnée par
		\[ (d) = \{ (x,y) \text{ tq. } x,y\in\R, \text{et }  y=3x+4 \} \] 
	et $(d')$ une droite parallèle à $(d)$ et passant par le point $(1;-4)$.
	Décrire $(d')$ : quelle équation tout point $(x,y)$ vérifie-t-il si et seulement s'il appartient à $(d')$ ?
}{
	La fonction $f(x) = ax+b$ associée à la droite $(d')$ vérifie 
		\[ a =3, \]
	par parallélisme, et
		\[ f(1) = -4, \]
	par appartenance du point $(1;-4)$.
	
	La première information donne directement $a$, et donc que $f(x) = 3x + b$, avec $b$ le paramètre restant à déterminer.
	La deuxième information donne alors
		\begin{align*}
			f(1) &= -4 \\
			3 + b &= -4 \\
			b = -7
		\end{align*}
	En conclusion, 
		\[ (d') = \{ (x,y) \text{ tq. } x,y\in\R, \text{et }  y=3x-7 \} \] 
}

\exe{
	Soient 
		\begin{align*}
		(d) &= \{ (x,y) \text{ tq. } x,y\in\R, \text{et }  y=3x+4 \} \\
		(d') &= \{ (x,y) \text{ tq. } x,y\in\R, \text{et }  y=-x-7 \}
		\end{align*}
	\begin{enumerate}
		\item Donner le point d'intersection des deux droite : l'élément de l'ensemble $(d) \cap (d')$.
		\item Donner les variations de $(d)$ et de $(d')$.
		\item Pour quelles valeurs de $x\in\R$ la droite $(d)$ est-elle au-dessus de $(d')$ ?	
	\end{enumerate}
}{
	\begin{enumerate}
		\item Considérons un point $P \in (d) \cap (d')$ appartenant aux deux droites.
		Notons $P(x_P; y_P)$ les coordonnées de $P$ qu'on souhaite trouver.
		
		Les appartenances donnent les équations suivantes.
			\[ \begin{cases*}
				y_P = 3x_P + 4, \\
				y_P = -x_P - 7
			\end{cases*} \]
		L'égalité $y_P = y_P$ étant évidemment vraie, on peut écrire
			\[ 3x_P + 4 = -x_P - 7, \]
		qui permet de résoudre pour $x_P$ calmement.
			\begin{align*}
				 3x_P + 4 &= -x_P - 7 \\
				 4x_P &= -11 \\
				 x_P &= -\dfrac{11}4
			\end{align*}
		Pour conclure, on trouve $y_P$ à l'aide d'une des deux équations du système.
		La deuxième équation, par exemple, donne
			\[ y_P = -x_P - 7 = \dfrac{11}4 - 7 = -\dfrac{17}4. \]
			
		Vérification : on vérifie que $P$ appartienne bien aux droites en calculant $3x_P + 4 = -\dfrac{33}4 + 4 = - \dfrac{17}4 = y_P$, et idem pour l'autre droite.
		\item $(d)$ est croissante et $(d')$ est décroissante.
		\item On résoud pour $x \in \R$ l'inéquation $3x+4 \geq -x-7$.
			\begin{align*}
				3x+4 &\geq -x - 7 \\
				4x &\geq -11 \\
				x &\geq -\dfrac{11}4
			\end{align*}
		En conclusion, la droite $(d)$ est au-dessus de $(d')$ pour tout $x\in \left[-\dfrac{11}4 ; +\infty\right[$.
		
		Il aurait été possible de conclure sans calculs : en effet, la droite $(d)$ étant croissante, et $(d')$ décroissante, la première domine l'autre pour toutes les valeurs de $x$ après $x_P$, l'abscisse du point d'intersection.
	\end{enumerate}
}


\end{document}