\documentclass[12pt]{paper}
\usepackage[french]{babel}
\usepackage[
a4paper,
margin=2cm,
nomarginpar,% We don't want any margin paragraphs
]{geometry}
\usepackage{fancyhdr}
\usepackage{array}
\usepackage{amsmath,amsfonts,amsthm,amssymb,mathtools,}
\newcolumntype{P}[1]{>{\centering\arraybackslash}p{#1}}


\usepackage{stackengine}
\newcommand\xrowht[2][0]{\addstackgap[.5\dimexpr#2\relax]{\vphantom{#1}}}

% theorems

\theoremstyle{plain}
\newtheorem{theorem}{Th\'eor\`eme}
\newtheorem*{sol}{Solution}
\theoremstyle{definition}
\newtheorem{ex}{Exercice}


% corps
\newcommand{\C}{\mathcal{C}}
\newcommand{\R}{\mathbb{R}}
\newcommand{\Rnn}{\mathbb{R}^{2n}}
\newcommand{\Z}{\mathbb{Z}}
\newcommand{\N}{\mathbb{N}}
\newcommand{\Q}{\mathbb{Q}}

% domain
\newcommand{\D}{\mathbb{D}}


% date
\usepackage{advdate}
\AdvanceDate[0]

% plots
\usepackage{tikz}
\usepackage{multicol}
\usepackage{pgfplots}

% for calligraphic C
\usepackage{calrsfs}

% euro
\usepackage{lmodern,textcomp}


% SOLUTION SWITCH
\newif\ifsolutions
				\solutionstrue
				\solutionsfalse

\ifsolutions
	\newcommand{\exe}[2]{
		\begin{ex} #1  \end{ex}
		\begin{sol} #2 \end{sol}
	}
\else
	\newcommand{\exe}[2]{
		\begin{ex} #1  \end{ex}
	}
	
\fi

\begin{document}
\pagestyle{fancy}
\fancyhead[L]{Première G2}
\fancyhead[C]{\textbf{Fonctions affines 2 \ifsolutions -- Solutions \fi}}
\fancyhead[R]{\today}

\exe{[Interpolation]
	Pour chacune des paires de points $A,B \in \R^2$ suivantes, calculer les paramètres de la fonction affine $f$ dont la courbe passe par ces deux points.
	
	\begin{multicols}{2}
	\begin{enumerate}
		\item $A(x_A;y_A), B(x_B; y_B)$.
		\item $A(2;8), B(4,7)$.
		\item $A(4;7), B(2;8)$.
		\item $A(-3; -3), B(-2; -1)$.
		\item $A(2;5), B(-10; 5)$.
		\item $A(-3;4), B(12;-11)$.
	\end{enumerate}
	\end{multicols}
}
{}

\exe{[Intersection]
	Pour chacune des paires de fonctions affines $f,g$, calculer l'intersection des droites $\C_f \cap \C_g$.
	
	\begin{multicols}{2}
	\begin{enumerate}
		\item $f(x) = 2x + 1, g(x) = -x+1$ $(x\in\R)$.
		\item $f(x) = -x + 1, g(x) = 2x + 1$ $(x\in\R)$.
		\item $f(x) = 3+7x, g(x) = 2$ $(x\in\R)$.
		\item $f(x) = 9-2x, g(x) = 17-x$ $(x\in\R)$.
		\item $f(x) = 2x+1, g(x) = 2x+1$ $(x\in\R)$.
		\item $f(x) = 2x+1, g(x) = 2x+2$ $(x\in\R)$.
	\end{enumerate}
	\end{multicols}
}{}

\exe{[Parallélisme]
	Pour chacun des couples de point $P$ et fonction affine $f$, trouver la fonction affine $g$ telle que $\C_f$ et $\C_g$ soient parallèles et que $P \in \C_g$.
	
	
	\begin{multicols}{2}
	\begin{enumerate}
		\item $P=(1;2)$ et $f(x) = 2x+1$ $(x\in\R)$.
		\item $P=(1;2)$ et $f(x) = 7-x$ $(x\in\R)$.
		\item $P=(1;2)$ et $f(x) = 1600$ $(x\in\R)$.
		\item $P=(1;2)$ et $f(x) = -3x + 5$ $(x\in\R)$.
	\end{enumerate}
	\end{multicols}
}{}


\exe{[Lecture graphique]

	Déterminer les paramètres des fonction affines $f,g,h$ dont les courbes sont représentées ci-dessous.

	\begin{center}
		\begin{tikzpicture}[>=stealth, scale=1.5]
		\begin{axis}[xmin = -10, xmax=10, ymin=-10, ymax=10, axis x line=middle, axis y line=middle, axis line style=<->, xlabel={}, ylabel={}, xtick = {-10, -8, ..., 8, 10}, ytick = {-10, -8, ..., 8, 10}, grid=both]
		
			\addplot[red, thick, domain =-9:9, samples=2] {-x}  node[above=6pt] {$(\mathcal{C}_f)$};
			\addplot[red, thick, dotted, domain =-10:-9, samples=2] {-x} ;
			\addplot[red, thick, dotted, domain =9:10, samples=2] {-x};
		
		
			\addplot[green, thick, domain =-9:9, samples=2] {x/2+1}  node[below=6pt] {$(\mathcal{C}_g)$};
			\addplot[green, thick, dotted, domain =-10:-9, samples=2] {x/2+1} ;
			\addplot[green, thick, dotted, domain =9:10, samples=2] {x/2+1};
		
		
			\addplot[black, thick, domain =-9:9, samples=2] {7}  node[above=10pt, left] {$(\mathcal{C}_h)$};
			\addplot[black, thick, dotted, domain =-10:-9, samples=2] {7} ;
			\addplot[black, thick, dotted, domain =9:10, samples=2] {7};
		
			
		\end{axis}
	\end{tikzpicture}
	\end{center}
}{}

\end{document}