\documentclass[12pt]{paper}
\usepackage[french]{babel}
\usepackage[
a4paper,
margin=2cm,
nomarginpar,% We don't want any margin paragraphs
]{geometry}
\usepackage{fancyhdr}
\usepackage{array}
\usepackage{amsmath,amsfonts,amsthm,amssymb,mathtools,enumitem}
\newcolumntype{P}[1]{>{\centering\arraybackslash}p{#1}}


\usepackage{stackengine}
\newcommand\xrowht[2][0]{\addstackgap[.5\dimexpr#2\relax]{\vphantom{#1}}}

% theorems

\theoremstyle{theorem}
\newtheorem{theorem}{Théorème}
\theoremstyle{definition}
\newtheorem{ex}{Exercice}


% corps
\newcommand{\C}{\mathcal{C}}
\newcommand{\R}{\mathbb{R}}
\newcommand{\Rnn}{\mathbb{R}^{2n}}
\newcommand{\Z}{\mathbb{Z}}
\newcommand{\N}{\mathbb{N}}
\newcommand{\Q}{\mathbb{Q}}

% domain
\newcommand{\D}{\mathbb{D}}


% date
\usepackage{advdate}
\AdvanceDate[0]

% plots
\usepackage{tikz}
\usepackage{multicol}
\usepackage{pgfplots}

% for calligraphic C
\usepackage{calrsfs}

% euro
\usepackage{lmodern,textcomp}

%commas...
\usepackage{icomma}


%tablestuff
\def\arraystretch{1.3}
\setlength\tabcolsep{50pt}

% SOLUTION SWITCH
\newif\ifsolutions
				\solutionstrue
				\solutionsfalse

\ifsolutions
	\newcommand{\exe}[2]{
		\begin{ex} #1  \end{ex}
		\begin{sol} #2 \end{sol}
	}
\else
	\newcommand{\exe}[2]{
		\begin{ex} #1  \end{ex}
	}
	
\fi

\begin{document}
\pagestyle{fancy}
\fancyhead[L]{Première G5}
\fancyhead[C]{\textbf{Évaluation : Fonctions affines \ifsolutions -- Solutions \fi}}
\fancyhead[R]{\today}


\begin{theorem}[Propriété fondamentale]\label{thm:1}
	Soit $f$ une fonction réelle sur $\R$ et $(x;y)\in\R^2$ un point du plan.
	Alors
		\begin{align*}
			(x ; y) \in \C_f && \iff && \underline{\qquad} = \underline{\qquad\qquad}.
		\end{align*}
\end{theorem}

\exe{[2pts]
	Compléter le théorème \ref{thm:1} vu en cours.
}{}


\exe{[2pts]
	Pour chaque fonction affine  sur $\R$ suivante, déterminer son coefficient directeur $a$ et son ordonnée à l'origine $b$.
	\begin{multicols}{2}
	\begin{enumerate}
		\item $f(x) = 10x + 2$
		\item $f(x) = 1 - x$
		\item $f(x) = -42$
	\end{enumerate}
	\end{multicols}
}{}


\exe{[2pts]
	Pour chacun des ensembles suivants, donner un élément lui appartenant.
		\begin{multicols}{2}
		\begin{enumerate}
			\item $\{ (x;y) \in \R^2 \text{ tels que } y=x \}$
			\item $\left\{ (x;y) \in \R^2 \text{ tq. } y = \dfrac{16}{7} x + 2 \right\}$
			\item $\{ (x;y) \in \R^2 \text{ tq. } y=-x \}$
			%\item $\{ (x;y) \in \R^2 \text{ tq. } y=42 \}$
		\end{enumerate}
		\end{multicols}
}{}

\exe{[4pts]
	Déterminer si les points suivants appartiennent ou non à $\C_f$ pour la fonction $f$ donnée par
		\[ f(x) =-1 - \dfrac23x \qquad \text{ pour tout } x\in\R. \]
	\begin{multicols}{2}
	\begin{enumerate}
		\item Le point $(-3;-3)$
		\item Le point $(3 ; -3)$
		\item Le point $(0;-1)$
	\end{enumerate}
	\end{multicols}
}{}

%\exe{[2pts]
%	Considérons une fonction affine $f$ quelconque avec $a,b \in \R$ ses paramètres :
%		\[ f(x) = ax + b \qquad \text{ pour tout } x\in\R. \]
%	Montrer $(0 ; b) \in \C_f$ et expliquer pourquoi $b$ s'appelle l'ordonnée à l'origine.
%}{}

\exe{[5pts]
    Achille dispute une course avec une tortue. On suppose que les deux participants avancent à vitesse constante. Posons $v = 60$ mètres par minute la vitesse d'Achille.
    La tortue, elle, avance 10 fois moins vite qu’Achille, soit à $6$ mètres par minute. 
    Achille décide donc de lui laisser généreusement 10 minutes d’avance.

    
	\begin{enumerate}
        \item Calculer la distance parcourue par la tortue après les 10 minutes laissées gracieusement par Achille.
        \item \`A la minute $10+t$, on appelle $A(t)$ et $T(t)$ les distances parcourues respectivement par Achille et la tortue, c'est-à-dire après les $10$ premières minutes accordées par Achille.
        Montrer que
            \begin{align*}
                A(t) = 60 t, && \text{ et } && T(t) = 60 + 6t.
            \end{align*}
        \item Combien de temps Achille met-il pour rattraper la tortue ?
        \item Quelle est la distance parcourue par les deux participants au moment où ils se croisent ?
        \item Supposons qu'on ne connaisse pas la vitesse exacte d'Achille mais qu'on sache tout de même que la tortue avance 10 fois moins vite que celui-ci.
        Montrer qu'Achille rattrape la tortue en $t = \dfrac{10}{9}$ minutes, indépendamment de la vitesse d'Achille.
	\end{enumerate}
	
}{}
\newpage

\exe{[2pts]
	Soit $f$ la fonction affine sur $\R$ donnée par
		\[ f(x) = -\frac15 x - 2^{64} \qquad \text{ pour tout } x\in\R. \]
    Déterminer l'expression algébrique de la fonction affine $h$ telle que $\C_h$ soit parallèle à $\C_g$ et passe par $(-10;5)$.
}{}

\exe{[3pts]
	Déterminer les paramètres (coefficient directeur $a$ et ordonnée à l'origine $b$) des fonction affines $f,h$ dont les courbes sont représentées ci-dessous.
	Les droites $\C_f$ et $\C_h$ sont parallèles.

	\begin{center}
		\begin{tikzpicture}[>=stealth, scale=1.5]
		\begin{axis}[xmin = -10, xmax=10, ymin=-10, ymax=10, axis x line=middle, axis y line=middle, axis line style=<->, xlabel={}, ylabel={}, xtick = {-10, -8, ..., 8, 10}, ytick = {-10, -8, ..., 8, 10}, grid=both]
		
			\addplot[red, thick, domain =-5:4, samples=2] {-2*x -1}  node[, pos=.15, left=4pt] {$\mathcal{C}_f$};
			\addplot[red, thick, dotted, domain =-6:-5, samples=2] {-2*x -1} ;
			\addplot[red, thick, dotted, domain =4:5, samples=2] {-2*x -1};
		
		
			%\addplot[olive, thick, domain =-3:8, samples=2] {3*x/2-4}  node[pos=.1, left=5pt] {$\mathcal{C}_g$};
			%\addplot[olive, thick, dotted, domain =-4:-3, samples=2] {3*x/2-4} ;
			%\addplot[olive, thick, dotted, domain =8:10, samples=2] {3*x/2-4};
		
		
			\addplot[black, thick, domain =0:8, samples=2] {-2*x + 8} node[above=4pt, pos=.25] {$\mathcal{C}_h$};
			\addplot[black, thick, dotted, domain =-2:0, samples=2] {-2*x + 8} ;
			\addplot[black, thick, dotted, domain =8:10, samples=2] {-2*x + 8};
		
			
		\end{axis}
	\end{tikzpicture}
	\end{center}
	
	
}{}



\subsection*{Bonus (3pts)}

\exe{
	Considérons une fonction quadratique 
		\[ f(x) = ax^2 + bx + c, \]
	où $a,b,c\in\R$ sont trois paramètres réels.
	Supposons de surcroît qu'on connaisse deux zéros distincts de $f$, c'est-à-dire qu'on connaisse $\alpha,\beta\in\R$ tels que $\alpha\neq\beta$ et
		\[ f(\alpha) = f(\beta) = 0. \]
	\begin{enumerate}
		\item Montrer que la fonction $g$ donnée par
			\[ g(x) = f(x) - a (x-\alpha)(x-\beta) \qquad \text{ pour tout } x\in\R \]
		est affine.
		\item Montrer que $g$ admet deux zéros distincts : $g(\alpha) = g(\beta) = 0$.
		\item En déduire, par interpolation linéaire, que $g$ est constamment nulle et donc que
			\[ f(x) = a (x-\alpha)(x-\beta)  \qquad \text{ pour tout } x\in\R.  \]
	\end{enumerate}
}

%\exe{
%	Le but de cet exercice est de montrer qu'une fonction quadratique se factorise en produit de fonctions affines.
%	Considérons, pour tout $x\in\R$,
%		\[ f(x) = x^2 -9x + 20. \]
%	\begin{enumerate}
%		\item Montrer que si $f(x) = (x-\alpha) (x-\beta)$ pour certains nombres réels $\alpha,\beta\in\R$, alors on a forcément $f(\alpha) = f(\beta) = 0$.
%	\end{enumerate}
%	Supposons désormais qu'il existe deux nombres distincts, $\alpha,\beta\in\R, \alpha\neq\beta$ tels que $f(\alpha)=f(\beta)=0$.
%	\begin{enumerate}
%		\item Montrer que
%			\[ g(x) = f(x) - (x-\alpha)(x-\beta) = (\alpha + \beta - 9) x + 20 - \alpha \beta. \]
%		\item En déduire que $g$ est affine et donner son coefficient directeur et son ordonnée à l'origine.
%		\item Montrer que $g(\alpha) = g(\beta)= 0$.
%		\item Par interpolation linéaire, montrer que $g$ est constamment nulle :
%			\[ g(x) = 0 \qquad \text{ pour tout } x\in\R. \]
%		\item En déduire que
%			\[\begin{cases} \alpha + \beta = 9, \\ \alpha \beta = 20. \end{cases} \]
%		\item Trouver $\alpha$ et $\beta$.
%	\end{enumerate}
%}{}

%\exe{[Bonus]
%	Soit $f(x) = ax+b$ une fonction affine sur $\R$ à paramètre $a,b\in\R$.
%	
%	Montrer que si $f(r)=0$ pour un certain $r\in\R$ on a alors, pour tout $x\in\R$,
%		\[ f(x) = a(x-r). \]
%}{}


\end{document}